\documentclass[preprint,12pt]{elsarticle}

\usepackage{amssymb}
\usepackage{url}
\usepackage[ruled,lined,linesnumbered,longend]{algorithm2e}

\journal{Computers and Operations Research}

\def\codemaker{CM}
\def\codebreaker{CB}
\begin{document}

\begin{frontmatter}

\title{An experimental study of exhaustive solutions for the Mastermind puzzle}

\author[ugr,citic]{Juan-J. Merelo-Guerv\'os}
\author[ugr,citic]{Antonio M. Mora}
\author[uma]{Carlos Cotta}
\author[hi]{Thomas P. Runarsson}
\address[ugr]{Dept. of Computer Architecture and Technology,
  University of Granada, Spain, \\ email: \{jmerelo,amorag\}@geneura.ugr.es}
\address[citic]{CITIC, \url{http://citic.ugr.es}}
\address[uma]{ Dept. of Languages and Computer
Sciences, University of M\'alaga, \\ email: ccottap@lcc.uma.es}
\address[hi]{School of Engineering and Natural Sciences, University of Iceland,\\ email: tpr@hi.is }


\end{frontmatter}

\section{Highlights}

This paper first presents a methodology for examining different
methods for solving Mastermind. This methodology focuses not only on
the search space reduction brough by each move, but also on the
probability of playing the secret combination by chance (not only by
reduction of the search space to one). Using this and applying it to
two different Mastermind instances, we find out the reasons why
different methods obtain their results and propose new methods that
are able to robustly obtain good results on a par with the
state-of-the-art methods. 

\end{document}
